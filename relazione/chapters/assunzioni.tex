\chapter{Assunzioni e Ipotesi}
\label{ch:assunzioni}

Dalle prime analisi effettuate sul dataset si sono riscontrati alcuni problemi rispetto alla gestione delle classi di qualità.\\
Come già descritto in precedenza sono presenti ben 10 differenti classi di qualità tra vini rossi e vini bianchi e questo porta a dover gestire un problema di classificazione a 10 classi.\\
Dopo aver osservato tramite l'analisi del dataset le varie distribuzioni dei dati si è scelto di raggruppare le classi di qualità portandosi in un problema di classificazione binario.\\
In questo capitolo verranno mostrati i grafici e spiegate le motivazioni che hanno portato questa scelta.\\
All'inizio sono state prese in considerazione tre possibili classificazioni:
\begin{itemize}
    \item \textbf{Classificazione a 10 classi}, ovvero quella originale rappresentata nel dataset.
    \item \textbf{Classificazione a 3 classi}, raggruppando le classificazioni originali dove i vini di qualità bassa sono costituiti dai vini con qualità tra 1 e 5, i vini di qualità media sono costituiti dai vini con qualità 6 e 7 e dove i vini di qualità alta sono costituiti dai vini con qualità 8 e 9.
    \item \textbf{Classificazione a 2 classi}, raggruppando le classificazioni originali dove i vini di qualità inferiore alla qualità originale 5 compresa sono vini di bassa qualità mentre i restanti fanno parte dei vini di alta qualità
\end{itemize}
Le tre tipologie sono state confrontate per poter scegliere quale risultasse la migliore, in primo luogo sono state osservate le distribuzioni rispetto alle istanze \ref{fig:quality_different_class}.\\

\begin{figure}
    \centering
    \includegraphics[scale=.23]{images/quality_different_class.jpg}
    \caption{grafico che rappresenta la distribuzione dei dati di vino bianco e di vino rosso rispetto alle tre tipologie di classificazioni prese in considerazione}
    \label{fig:quality_different_class}
\end{figure}

Da questa analisi si è notato come le istanze non siano distribuite in modo uniforme, anzi si ha una prevalenza di dati rispetto alle qualità centrali e una scarsa rappresentazione delle qualità più basse e più alte, per questo motivo si è scartata l'ipotesi di poter sfruttare una classificazione a 10 classi, perchè non in grado di rappresentare in modo coerente le 10 classi.\\
Per scegliere con maggiore certezza tra le differenti tipologie si è scesi nel dettaglio cercando tramite un grafico \ref{fig:quality_different_class_different_class} che rappresentasse maggiormente le differenze di distribuzione delle istanze, questo grafico rappresenta come il precedente la distribuzione delle istanze rispetto alle tipologie di classificazioni in analisi aggiungendo in più la possibilità di poter effettuare un confronto anche tra vino rosso e vino bianco.

\begin{figure}
    \centering
    \includegraphics[scale=.23]{images/quality_different_class_different_type.jpg}
    \caption{grafico che rappresenta la distribuzione dei dati di vino bianco e di vino rosso in modo separato considerando le tre tipologie di classificazioni prese in considerazione}
    \label{fig:quality_different_class_different_class}
\end{figure}

\noindent
Osservando i due grafici \ref{fig:quality_different_class} e \ref{fig:quality_different_class_different_class} si può notare ulteriormente come per alcune classi di qualità non siano presenti istanze che li rappresentino.\\
La classificazione a 3 classi risolve solo parzialmente il problema della versione a 10 classi perchè per le qualità più alte si mantiene una bassa rappresentazione rispetto al numero di istanze presenti nelle altre due categorie di qualità.\\
La classificazione a due classi è stata scelta perchè anche se sbilanciata ha un buon numero di istanze che rappresentano sia la qualità bassa si la qualità alta.
