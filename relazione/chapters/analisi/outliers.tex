\section{Analisi Outlier}
Gli outlier sono dei valori anomali o estremi, lontani dai valori centrali di un insieme di dati. Questi valori influenzano negativamente la media e la deviazione standard del dataset e quindi possono portare a risultati sbagliati. Molti algoritmi di machine learning non funzionano in modo ottimale in presenza di outlier e quindi c'è bisogno di rilevarli e rimuoverli.

\vspace{4mm}
\noindent
\`{E} stata effettuata una ricerca degli outlier su ogni attributo numerico attraverso i seguenti metodi statistici:

\subsubsection{IQR}
Gli outlier sono stati individuati usando l'approccio basato sul Interquartile Range (IQR). Lo scarto interquartile è un indice di dispersione, ovvero una misura di quanto i valori si allontanino da un valore centrale. Viene calcolato dalla differenza tra il terzo quartile (Q3) e il primo quartile (Q1). In questo approccio tutti i punti che si trovano al di sopra del valore Q3 + 1.5 * IQR o al di sotto del valore Q1 - 1.5 * IQR sono considerati outlier.

\begin{align*}
    IQR         & = Q3 - Q1        \\
    Lower Bound & = Q1 - 1.5 * IQR \\
    Upper Bound & = Q3 + 1.5 * IQR
\end{align*}

\noindent
Gli outlier possono essere rimossi o sostituiti con un valore fissato come ad esempio media, moda, mediana.
Dal momento che il dataset è sbilanciato, l'opzione di rimuovere completamente gli outlier è stata scartata perché molti di questi appartengono alla classe minoritaria.
In questo lavoro si è scelto quindi di sostituire i valori con la mediana, poiché alcune variabili hanno distribuzioni con una distorsione unilaterale e quindi corrisponde a un valore più vicino al centro rispetto alla media.

\begin{figure}
    \centering
    \includegraphics[width=.8\textwidth]{images/IQR.png}
    \caption{Esempio di scarto interquartile in una distribuzione normale \cite{wikipedia:iqr}}
    \label{fig:iqr}
\end{figure}

\subsubsection{Winsorizing (Percentile Capping)}
E' un metodo simile al metodo IQR, in questo caso si utilizzano due percentili. Tutti i valori sotto al minimo valore dell'intervallo vengono sostituiti con il minimo, e tutti i valori sopra il massimo valore dell'intervallo vengono sostituiti con il massimo. In questo lavoro sono stati usati due intervalli ($5^{\circ}$ percentile, $95^{\circ}$ percentile) e ($1^{\circ}$ percentile, $99^{\circ}$ percentile).

\noindent
I due intervalli sono stati denominati Winsorizing 90\% e Winsorizing 98\%:
\begin{itemize}
    \item Winsorizing 90\% indica che il 5\% inferiore dei dati viene sostituito con il $5^{\circ}$ percentile e il 5\% superiore dei dati viene sostituito con il $95^{\circ}$ percentile.
    \item Winsorizing 98\% indica che l' 1\% inferiore dei dati viene sostituito con il $1^{\circ}$ percentile e l' 1\% superiore dei dati viene sostituito con il $99^{\circ}$ percentile.
\end{itemize}

\subsubsection{Metodo Scelto}
Un grafico abbastanza semplice e veloce per visualizzare gli outlier è il boxplot. \'{E} stato confrontato ogni variabile del dataset con i valori assunti dopo l'applicazione dei metodi di rimozione degli outlier (IQR, Winsorizing 90\% e Winsorizing 98\%), attraverso dei boxplot. Sopra a ogni boxplot sono stati riportati i valori divisi per qualità.

\noindent
Il metodo Winsorizing rileva un intervallo di outlier più piccolo e variabile rispetto all'IQR. Inoltre nei casi di distribuzione con distorsione laterale accumula troppi valori agli estremi, alterando così la distribuzione. Con il metodo Winsorizing 98\% si risulta avere una distribuzione più smussata agli estremi. Il metodo IQR, sostituendo con la mediana non altera molto la distribuzione.
Per decidere il metodo più efficace da usare sono stati confrontati dati dopo l'applicazione di ogni metodo attraverso dei Q-Q plot.

\noindent
I Q-Q (quantile-quantile) plot sono dei grafici utili per capire se due insiemi di dati hanno la stessa distribuzione. Vengono rappresentati i punti in un piano cartesiano attraverso una coppia di quantili. Inoltre viene tracciata una retta a 45° in modo da evidenziare i punti più vicini alla retta. Due insiemi di dati hanno una distribuzione simile se i punti cadono approssimatamene sulla linea di riferimento.
Analizzando i grafici si è visto che il metodo IQR ha valori più vicini alla retta, quindi si è scelto di utilizzare questo.

\noindent
Nelle seguenti tabelle sono stati riportati le varie statistiche descrittive delle variabili prima e dopo la rimozione degli outlier con il metodo scelto.

\begin{table}
\centering
\resizebox{\linewidth}{!}{
\begin{tabular}[t]{lrrrrrrrr}
\toprule
  & vars & mean & sd & median & min & max & skew & kurtosis\\
\midrule
\cellcolor{gray!6}{fixed.acidity} & \cellcolor{gray!6}{1} & \cellcolor{gray!6}{6.86} & \cellcolor{gray!6}{0.85} & \cellcolor{gray!6}{6.80} & \cellcolor{gray!6}{3.80} & \cellcolor{gray!6}{14.20} & \cellcolor{gray!6}{0.71} & \cellcolor{gray!6}{2.67}\\
volatile.acidity & 2 & 0.28 & 0.10 & 0.26 & 0.08 & 1.10 & 1.71 & 6.18\\
\cellcolor{gray!6}{citric.acid} & \cellcolor{gray!6}{3} & \cellcolor{gray!6}{0.33} & \cellcolor{gray!6}{0.12} & \cellcolor{gray!6}{0.32} & \cellcolor{gray!6}{0.00} & \cellcolor{gray!6}{1.66} & \cellcolor{gray!6}{1.40} & \cellcolor{gray!6}{7.37}\\
residual.sugar & 4 & 6.45 & 5.13 & 5.20 & 0.60 & 65.80 & 1.20 & 4.87\\
\cellcolor{gray!6}{chlorides} & \cellcolor{gray!6}{5} & \cellcolor{gray!6}{0.05} & \cellcolor{gray!6}{0.02} & \cellcolor{gray!6}{0.04} & \cellcolor{gray!6}{0.01} & \cellcolor{gray!6}{0.35} & \cellcolor{gray!6}{5.31} & \cellcolor{gray!6}{42.44}\\
\addlinespace
free.sulfur.dioxide & 6 & 35.33 & 17.24 & 34.00 & 2.00 & 289.00 & 1.71 & 15.25\\
\cellcolor{gray!6}{total.sulfur.dioxide} & \cellcolor{gray!6}{7} & \cellcolor{gray!6}{138.62} & \cellcolor{gray!6}{42.35} & \cellcolor{gray!6}{135.00} & \cellcolor{gray!6}{9.00} & \cellcolor{gray!6}{440.00} & \cellcolor{gray!6}{0.41} & \cellcolor{gray!6}{0.71}\\
density & 8 & 0.99 & 0.00 & 0.99 & 0.99 & 1.04 & 1.22 & 13.31\\
\cellcolor{gray!6}{pH} & \cellcolor{gray!6}{9} & \cellcolor{gray!6}{3.19} & \cellcolor{gray!6}{0.15} & \cellcolor{gray!6}{3.18} & \cellcolor{gray!6}{2.74} & \cellcolor{gray!6}{3.82} & \cellcolor{gray!6}{0.47} & \cellcolor{gray!6}{0.52}\\
sulphates & 10 & 0.49 & 0.11 & 0.48 & 0.23 & 1.08 & 1.00 & 1.68\\
\addlinespace
\cellcolor{gray!6}{alcohol} & \cellcolor{gray!6}{11} & \cellcolor{gray!6}{10.52} & \cellcolor{gray!6}{1.24} & \cellcolor{gray!6}{10.40} & \cellcolor{gray!6}{8.00} & \cellcolor{gray!6}{14.20} & \cellcolor{gray!6}{0.49} & \cellcolor{gray!6}{-0.69}\\
\bottomrule
\end{tabular}}
\end{table}


\begin{table}[H]
\centering
\resizebox{\linewidth}{!}{
\begin{tabular}[t]{lrrrrrrrr}
\toprule
  & vars & mean & sd & median & min & max & skew & kurtosis\\
\midrule
\cellcolor{gray!6}{fixed.acidity} & \cellcolor{gray!6}{1} & \cellcolor{gray!6}{8.21} & \cellcolor{gray!6}{1.59} & \cellcolor{gray!6}{7.90} & \cellcolor{gray!6}{4.70} & \cellcolor{gray!6}{12.60} & \cellcolor{gray!6}{0.63} & \cellcolor{gray!6}{-0.08}\\
volatile.acidity & 2 & 0.52 & 0.17 & 0.51 & 0.12 & 1.00 & 0.27 & -0.30\\
\cellcolor{gray!6}{citric.acid} & \cellcolor{gray!6}{3} & \cellcolor{gray!6}{0.27} & \cellcolor{gray!6}{0.19} & \cellcolor{gray!6}{0.26} & \cellcolor{gray!6}{0.00} & \cellcolor{gray!6}{0.79} & \cellcolor{gray!6}{0.29} & \cellcolor{gray!6}{-0.89}\\
residual.sugar & 4 & 2.17 & 0.45 & 2.10 & 0.90 & 3.60 & 0.56 & 0.42\\
\cellcolor{gray!6}{chlorides} & \cellcolor{gray!6}{5} & \cellcolor{gray!6}{0.08} & \cellcolor{gray!6}{0.01} & \cellcolor{gray!6}{0.08} & \cellcolor{gray!6}{0.04} & \cellcolor{gray!6}{0.12} & \cellcolor{gray!6}{0.19} & \cellcolor{gray!6}{0.12}\\
\addlinespace
free.sulfur.dioxide & 6 & 15.11 & 9.33 & 13.00 & 1.00 & 42.00 & 0.79 & -0.21\\
\cellcolor{gray!6}{total.sulfur.dioxide} & \cellcolor{gray!6}{7} & \cellcolor{gray!6}{41.46} & \cellcolor{gray!6}{26.04} & \cellcolor{gray!6}{35.00} & \cellcolor{gray!6}{6.00} & \cellcolor{gray!6}{116.00} & \cellcolor{gray!6}{0.88} & \cellcolor{gray!6}{-0.06}\\
density & 8 & 1.00 & 0.00 & 1.00 & 0.99 & 1.00 & 0.01 & -0.10\\
\cellcolor{gray!6}{pH} & \cellcolor{gray!6}{9} & \cellcolor{gray!6}{3.31} & \cellcolor{gray!6}{0.15} & \cellcolor{gray!6}{3.31} & \cellcolor{gray!6}{2.92} & \cellcolor{gray!6}{3.71} & \cellcolor{gray!6}{0.02} & \cellcolor{gray!6}{-0.11}\\
sulphates & 10 & 0.64 & 0.12 & 0.62 & 0.33 & 0.99 & 0.55 & -0.10\\
\addlinespace
\cellcolor{gray!6}{alcohol} & \cellcolor{gray!6}{11} & \cellcolor{gray!6}{10.42} & \cellcolor{gray!6}{1.03} & \cellcolor{gray!6}{10.20} & \cellcolor{gray!6}{8.40} & \cellcolor{gray!6}{13.50} & \cellcolor{gray!6}{0.72} & \cellcolor{gray!6}{-0.33}\\
\bottomrule
\end{tabular}}
\caption{Dopo la rimozione con IQR}
\end{table}


\subsection{Grafici}

\begin{figure}[H]
    \centering

    \subfloat[]{%
        \includegraphics[width=0.99\textwidth]{images/outliers/alcohol_boxplot.png}
    }

    \subfloat[]{%
        \includegraphics[width=0.45\textwidth]{images/outliers/alcohol_distribution.png}
    }\qquad
    \subfloat[]{%
        \includegraphics[width=0.45\textwidth]{images/outliers/alcohol_qqplot.png}
    }

    \label{fig:alcohol}
    \caption{Commento}
\end{figure}

\begin{figure}[H]
    \centering

    \subfloat[]{%
        \includegraphics[width=0.99\textwidth]{images/outliers/chlorides_boxplot.png}
    }

    \subfloat[]{%
        \includegraphics[width=0.45\textwidth]{images/outliers/chlorides_distribution.png}
    }\qquad
    \subfloat[]{%
        \includegraphics[width=0.45\textwidth]{images/outliers/chlorides_qqplot.png}
    }

    \label{fig:chlorides}
    \caption{Commento}
\end{figure}

\begin{figure}[H]
    \centering

    \subfloat[]{%
        \includegraphics[width=0.99\textwidth]{images/outliers/citric.acid_boxplot.png}
    }

    \subfloat[]{%
        \includegraphics[width=0.45\textwidth]{images/outliers/citric.acid_distribution.png}
    }\qquad
    \subfloat[]{%
        \includegraphics[width=0.45\textwidth]{images/outliers/citric.acid_qqplot.png}
    }

    \label{fig:citric.acid}
    \caption{Commento}
\end{figure}

\begin{figure}[H]
    \centering

    \subfloat[]{%
        \includegraphics[width=0.99\textwidth]{images/outliers/density_boxplot.png}
    }

    \subfloat[]{%
        \includegraphics[width=0.45\textwidth]{images/outliers/density_distribution.png}
    }\qquad
    \subfloat[]{%
        \includegraphics[width=0.45\textwidth]{images/outliers/density_qqplot.png}
    }

    \label{fig:density}
    \caption{Commento}
\end{figure}

\begin{figure}[H]
    \centering

    \subfloat[]{%
        \includegraphics[width=0.99\textwidth]{images/outliers/fixed.acidity_boxplot.png}
    }

    \subfloat[]{%
        \includegraphics[width=0.45\textwidth]{images/outliers/fixed.acidity_distribution.png}
    }\qquad
    \subfloat[]{%
        \includegraphics[width=0.45\textwidth]{images/outliers/fixed.acidity_qqplot.png}
    }

    \label{fig:fixed.acidity}
    \caption{Commento}
\end{figure}

\begin{figure}[H]
    \centering

    \subfloat[]{%
        \includegraphics[width=0.99\textwidth]{images/outliers/free.sulfur.dioxide_boxplot.png}
    }

    \subfloat[]{%
        \includegraphics[width=0.45\textwidth]{images/outliers/free.sulfur.dioxide_distribution.png}
    }\qquad
    \subfloat[]{%
        \includegraphics[width=0.45\textwidth]{images/outliers/free.sulfur.dioxide_qqplot.png}
    }

    \label{fig:free.sulfur.dioxide}
    \caption{Commento}
\end{figure}

\begin{figure}[H]
    \centering

    \subfloat[]{%
        \includegraphics[width=0.99\textwidth]{images/outliers/pH_boxplot.png}
    }

    \subfloat[]{%
        \includegraphics[width=0.45\textwidth]{images/outliers/pH_distribution.png}
    }\qquad
    \subfloat[]{%
        \includegraphics[width=0.45\textwidth]{images/outliers/pH_qqplot.png}
    }

    \label{fig:pH}
    \caption{Commento}
\end{figure}

\begin{figure}[H]
    \centering

    \subfloat[]{%
        \includegraphics[width=0.99\textwidth]{images/outliers/residual.sugar_boxplot.png}
    }

    \subfloat[]{%
        \includegraphics[width=0.45\textwidth]{images/outliers/residual.sugar_distribution.png}
    }\qquad
    \subfloat[]{%
        \includegraphics[width=0.45\textwidth]{images/outliers/residual.sugar_qqplot.png}
    }

    \label{fig:residual.sugar}
    \caption{Commento}
\end{figure}

\begin{figure}[H]
    \centering

    \subfloat[]{%
        \includegraphics[width=0.99\textwidth]{images/outliers/sulphates_boxplot.png}
    }

    \subfloat[]{%
        \includegraphics[width=0.45\textwidth]{images/outliers/sulphates_distribution.png}
    }\qquad
    \subfloat[]{%
        \includegraphics[width=0.45\textwidth]{images/outliers/sulphates_qqplot.png}
    }

    \label{fig:sulphates}
    \caption{Commento}
\end{figure}

\begin{figure}[H]
    \centering

    \subfloat[]{%
        \includegraphics[width=0.99\textwidth]{images/outliers/total.sulfur.dioxide_boxplot.png}
    }

    \subfloat[]{%
        \includegraphics[width=0.45\textwidth]{images/outliers/total.sulfur.dioxide_distribution.png}
    }\qquad
    \subfloat[]{%
        \includegraphics[width=0.45\textwidth]{images/outliers/total.sulfur.dioxide_qqplot.png}
    }

    \label{fig:total.sulfur.dioxide}
    \caption{Commento}
\end{figure}

\begin{figure}[H]
    \centering

    \subfloat[]{%
        \includegraphics[width=0.99\textwidth]{images/outliers/volatile.acidity_boxplot.png}
    }

    \subfloat[]{%
        \includegraphics[width=0.45\textwidth]{images/outliers/volatile.acidity_distribution.png}
    }\qquad
    \subfloat[]{%
        \includegraphics[width=0.45\textwidth]{images/outliers/volatile.acidity_qqplot.png}
    }

    \label{fig:volatile.acidity}
    \caption{Commento}
\end{figure}
